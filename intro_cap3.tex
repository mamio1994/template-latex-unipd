Per inserire delle formule matematiche è possibile utilizzare due metodi:
\begin{itemize}
\item in linea: inserendo la formula tra due caratteri \$. 
\item utilizzando l'ambiente \textit{equation}.
\end{itemize}

%Il comando noindent viene utilizzato per evitare l'indentazione del paragrafo
\noindent Esempio di formula in linea $x(t)=x_{0} + v_{0}t + \frac{1}{2}at^{2}$.

\noindent Esempio di utilizzo dell'ambiente \textit{equation}:
\begin{equation}\label{eq:legge_oraria}
x(t)=x_{0} + v_{0}t + \frac{1}{2}at^{2}
\end{equation}
Possiamo usare le \textit{label} anche per le equazioni. Legge oraria nell'Equazione \ref{eq:legge_oraria}.
Infine, un esempio di formula su più righe:
\begin{equation}
\begin{split}
x(t) &= x_{0} + v_{0}t + \frac{1}{2}at^{2} \\ & =
		x_{0} + v_{0}t + \frac{1}{2}\frac{F}{m}t^{2}
\end{split}
\end{equation}